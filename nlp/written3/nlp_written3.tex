%    2. Write your answers in section "B" below. Precede answers for all 
%       parts of a question with the command "\question{n}{desc}" where n is
%       the question number and "desc" is a short, one-line description of 
%       the problem. There is no need to restate the problem.
%    3. If a question has multiple parts, precede the answer to part x with the
%       command "\part{x}".
%    4. If a problem asks you to design an algorithm, use the commands
%       \algorithm, \correctness, \runtime to precede your discussion of the 
%       description of the algorithm, its correctness, and its running time, respectively.
%    5. You can include graphics by using the command \includegraphics{FILENAME}
%
\documentclass[11pt]{article}
\usepackage{amsmath,amssymb,amsthm}
\usepackage{graphicx}
\usepackage[margin=1in]{geometry}
\usepackage{fancyhdr}
\usepackage{float}
\setlength{\parindent}{0pt}
\setlength{\parskip}{5pt plus 1pt}
\setlength{\headheight}{13.6pt}
\newcommand\question[2]{\vspace{.25in}\hrule\textbf{#1: #2}\vspace{.5em}\hrule\vspace{.10in}}
\renewcommand\part[1]{\vspace{.10in}\textbf{(#1)}}
\pagestyle{fancyplain}
\lhead{\textbf{\NAME\ (\UID)}}
\chead{\textbf{HW\HWNUM}}
\rhead{CS 5340, \today}
\begin{document}\raggedright

\newcommand\NAME{Jake Pitkin}
\newcommand\UID{u0891770}
\newcommand\HWNUM{3}

\question{Problem 1}{Thematic Role}

 \begin{table}[H]
\centering
{\renewcommand{\arraystretch}{1.2}%
\begin{tabular}{| c | c | c |}
\hline
\textbf{Question} & \textbf{Noun Phrase} & \textbf{Thematic Role}\\
\hline
a & John & agent\\ \hline
a & the sidewalk & theme\\ \hline
a & snow shovel & instrument\\ \hline
b & A raffle & theme\\ \hline
b & the local charity & beneficiary\\ \hline
c & Joe & agent\\ \hline
c & Susan & co-agent\\ \hline
d & Tom & recipient\\ \hline
d & a bike & theme\\ \hline
e & The dress & theme\\ \hline
e & Lisa & agent\\ \hline
e & matching shoes & co-theme\\ \hline
f & Julie & agent\\ \hline
f & grandmother & recipient\\ \hline
f & a letter & theme\\ \hline
g & George & experiencer\\ \hline
g & car & theme\\ \hline
h & The image & theme\\ \hline
h & Adobe Photoshop & instrument\\ \hline
h & cartoonist & agent\\ \hline
\end{tabular}}
\end{table}

\question{Problem 2}{Similarities Between Vectors}

\part{a}
$$ManhattanDistance(x, y) = \sum_{i = 1}^N |x_i - y_i| = 3 + 5 + 4 + 5 = 17$$

\framebox[1.2\width]{Manhattan Distance of 17}

\part{b}
$$ManhattanDistance(x, y) = \sum_{i = 1}^N |x_i - y_i| = 5 + 4 + 5 + 3 = 17$$

\framebox[1.2\width]{Manhattan Distance of 17}

\part{c}
$$Jaccard(x, y) = \frac{\sum_{i = 1}^N min(x_i, y_i)}{\sum_{i = 1}^N max(x_i, y_i)} = \frac{6 + 2 + 4 + 0}{9 + 7 + 8 + 5} = \frac{12}{29}$$

\framebox[1.2\width]{Jaccard Similarity of $\frac{12}{29}$}

\part{d}
$$Jaccard(x, y) = \frac{\sum_{i = 1}^N min(x_i, y_i)}{\sum_{i = 1}^N max(x_i, y_i)} = \frac{1 + 3 + 4 + 2}{6 + 7 + 9 + 5} = \frac{10}{27}$$

\framebox[1.2\width]{Jaccard Similarity of $\frac{10}{27}$}

\part{e}
$$Cosine(x, y) = \frac{\sum_{i = 1}^N (x_i * y_i)}{\sqrt{\sum_{i = 1}^N x_i^2}\sqrt{\sum_{i = i}^N y_i^2}} = \frac{54 + 14 + 32 + 0}{\sqrt{81 + 4 + 64 + 0}*\sqrt{36 + 49 + 16 + 25}} = 0.73$$

\framebox[1.2\width]{Cosine Similarity of 0.73}

\part{f}
$$Cosine(x, y) = \frac{\sum_{i = 1}^N (x_i * y_i)}{\sqrt{\sum_{i = 1}^N x_i^2}\sqrt{\sum_{i = i}^N y_i^2}} = \frac{6 + 21 + 36 + 10}{\sqrt{36 + 49 + 16 + 25}*\sqrt{1 + 9 + 81 + 4}} = 0.67$$

\framebox[1.2\width]{Cosine Similarity of 0.67}

\question{Problem 3}{Collins \& Singer}

\part{a}

 \begin{table}[H]
\centering
{\renewcommand{\arraystretch}{1.4}%
\begin{tabular}{| c | c |}
\hline
\textbf{Rule} & \textbf{Probability}\\
\hline
\textbf{If Contains(}apple\textbf{)} $\rightarrow$ {\tt PRODUCT} & 1/3\\ \hline
\textbf{If Contains(}apple\textbf{)} $\rightarrow$ {\tt COMPANY} & 2/3\\ \hline
\textbf{If Contains(}tablet\textbf{)} $\rightarrow$ {\tt PRODUCT} & 3/4\\ \hline
\textbf{If Contains(}tablet\textbf{)} $\rightarrow$ {\tt COMPANY} & 1/4\\ \hline
\textbf{If Contains(}microsoft\textbf{)} $\rightarrow$ {\tt PRODUCT} & 1/3\\ \hline
\textbf{If Contains(}microsoft\textbf{)} $\rightarrow$ {\tt COMPANY} & 2/3\\ \hline
\textbf{If Contains(}british\textbf{)} $\rightarrow$ {\tt PRODUCT} & 1/2\\ \hline
\textbf{If Contains(}british\textbf{)} $\rightarrow$ {\tt COMPANY} & 1/2\\ \hline
\textbf{If Contains(}corporation\textbf{)} $\rightarrow$ {\tt PRODUCT} & 0/3\\ \hline
\textbf{If Contains(}corporation\textbf{)} $\rightarrow$ {\tt COMPANY} & 3/3\\ \hline
\end{tabular}}
\end{table}

\part{b}

 \begin{table}[H]
\centering
{\renewcommand{\arraystretch}{1.4}%
\begin{tabular}{| c | c |}
\hline
\textbf{Rule} & \textbf{Probability}\\
\hline
\textbf{If Contains(}mobile\textbf{)} $\rightarrow$ {\tt PRODUCT} & 2/3\\ \hline
\textbf{If Contains(}mobile\textbf{)} $\rightarrow$ {\tt COMPANY} & 1/3\\ \hline
\textbf{If Contains(}computer\textbf{)} $\rightarrow$ {\tt PRODUCT} & 3/4\\ \hline
\textbf{If Contains(}computer\textbf{)} $\rightarrow$ {\tt COMPANY} & 1/4\\ \hline
\textbf{If Contains(}tech\textbf{)} $\rightarrow$ {\tt PRODUCT} & 1/4\\ \hline
\textbf{If Contains(}tech\textbf{)} $\rightarrow$ {\tt COMPANY} & 3/4\\ \hline
\textbf{If Contains(}giant\textbf{)} $\rightarrow$ {\tt PRODUCT} & 0/3\\ \hline
\textbf{If Contains(}giant\textbf{)} $\rightarrow$ {\tt COMPANY} & 3/3\\ \hline
\textbf{If Contains(}leader\textbf{)} $\rightarrow$ {\tt PRODUCT} & 0/2\\ \hline
\textbf{If Contains(}leader\textbf{)} $\rightarrow$ {\tt COMPANY} & 2/2\\ \hline
\end{tabular}}
\end{table}

\question{Problem 4}{Salience Values}
There are four sentences with a {\tt FISH} context: \textbf{S1, S5, S6,} and \textbf{S7.}

There are five sentences with a {\tt MUSIC} context: \textbf{S2, S3, S4, S5,} and \textbf{S6.}

Note that \textbf{S5} and \textbf{S6} have both a {\tt FISH} and {\tt MUSIC} context.

\part{a} \textit{Utah} appears in three sentences with a {\tt FISH} context: \textbf{S1, S6,} and \textbf{S7}. \textit{Utah} appears 4 times in the corpus.

$$salience(Utah, FISH) = \frac{P(Utah | FISH)}{P(Utah)} = \frac{3/4}{4/60} = 11.25$$

\framebox[1.2\width]{Salience value of 11.25.}

\part{b} \textit{Electric} appears in two sentences with a {\tt FISH} context: \textbf{S5} and \textbf{S6}. \textit{Electric} appears 3 times in the corpus.

$$salience(electric, FISH) = \frac{P(electric | FISH)}{P(electric)} = \frac{2/4}{3/60} = 10$$

\framebox[1.2\width]{Salience value of 10.}

\part{c} \textit{Bass} appears in two sentences with a {\tt FISH} context: \textbf{S5} and \textbf{S6}. \textit{Bass} appears 5 times in the corpus.

$$salience(bass, FISH) = \frac{P(bass | FISH)}{P(bass)} = \frac{2/4}{5/60} = 6$$

\framebox[1.2\width]{Salience value of 6.}

\part{d} \textit{Utah} appears in two sentences with a {\tt MUSIC} context: \textbf{S2} and \textbf{S6}. \textit{Utah} appears 4 times in the corpus.

$$salience(Utah, MUSIC) = \frac{P(Utah | MUSIC)}{P(Utah)} = \frac{2/5}{4/60} = 6$$

\framebox[1.2\width]{Salience value of 6.}

\part{e} \textit{Electric} appears in three sentences with a {\tt MUSIC} context: \textbf{S3, S5} and \textbf{S6}. \textit{Electric} appears 3 times in the corpus.

$$salience(electric, MUSIC) = \frac{P(electric | MUSIC)}{P(electric)} = \frac{3/5}{3/60} = 12$$

\framebox[1.2\width]{Salience value of 12.}

\part{f} \textit{Bass} appears in five sentences with a {\tt MUSIC} context: \textbf{S2, S3, S4, S5} and \textbf{S6}. \textit{Bass} appears 5 times in the corpus.

$$salience(bass, MUSIC) = \frac{P(bass | MUSIC)}{P(bass)} = \frac{5/5}{5/60} = 12$$

\framebox[1.2\width]{Salience value of 12.}

\question{Problem 5}{Antecedents and Pronouns}

\part{a}
\begin{itemize}
  \item John Smith
  \item John
  \item his
  \item him
\end{itemize}

\part{b}
\begin{itemize}
  \item assuming "them" is ANIMATE
  \item John Smith
  \item John
  \item Mary
  \item his
  \item She
  \item him
  \item He
  \item he
\end{itemize}

\part{c}
\begin{itemize}
  \item 10 oranges
  \item too many groceries	
\end{itemize}

\part{d}
\begin{itemize}
  \item himself 	
\end{itemize}

\part{e}
\begin{itemize}
  \item his
  \item his
  \item their
  \item her 	
\end{itemize}

\part{f}
\begin{itemize}
  \item It
  \item it 	
\end{itemize}

\part{g}
\begin{itemize}
  \item his neighbor
  \item George 	
\end{itemize}

\end{document}
