%    2. Write your answers in section "B" below. Precede answers for all 
%       parts of a question with the command "\question{n}{desc}" where n is
%       the question number and "desc" is a short, one-line description of 
%       the problem. There is no need to restate the problem.
%    3. If a question has multiple parts, precede the answer to part x with the
%       command "\part{x}".
%    4. If a problem asks you to design an algorithm, use the commands
%       \algorithm, \correctness, \runtime to precede your discussion of the 
%       description of the algorithm, its correctness, and its running time, respectively.
%    5. You can include graphics by using the command \includegraphics{FILENAME}
%
\documentclass[11pt]{article}
\usepackage{amsmath,amssymb,amsthm}
\usepackage{graphicx}
\usepackage[margin=1in]{geometry}
\usepackage{fancyhdr}
\setlength{\parindent}{0pt}
\setlength{\parskip}{5pt plus 1pt}
\setlength{\headheight}{13.6pt}
\newcommand\question[2]{\vspace{.25in}\hrule\textbf{#1: #2}\vspace{.5em}\hrule\vspace{.10in}}
\renewcommand\part[1]{\vspace{.10in}\textbf{(#1)}}
\newcommand\algorithm{\vspace{.10in}\textbf{Algorithm: }}
\newcommand\correctness{\vspace{.10in}\textbf{Correctness: }}
\newcommand\runtime{\vspace{.10in}\textbf{Running time: }}
\pagestyle{fancyplain}
\lhead{\textbf{\NAME\ (\UID)}}
\chead{\textbf{HW\HWNUM}}
\rhead{CS 6150, \today}
\begin{document}\raggedright

\newcommand\NAME{Jake Pitkin}
\newcommand\UID{u0891770}
\newcommand\HWNUM{0}

\question{1}{Big Oh}
\part{a} $O(n^2)$
% leaving an empty line between the parts makes the output look better

\part{b} $O(\log n)$

\part{c} $O(1/n)$

\part{d} $O(1)$

\part{e} $O(\log n)$

\question{2}{Removing Duplicates} 

\algorithm: First, sort the array A (using merge sort), we will call this sorted array C which is of size \emph{n}.  Next, iterate through C, in order, while keeping track of the previous element. If the current element is not equal to the previous element (or it's the first element), add it to B. Else, continue to the next element.

\correctness: We can observe that every distinct element in C is included in B. Since A and C contain the same elements, we have correctness.

\runtime: The step of sorting A with merge sort take $O(\log n)$ time and the step of iterating B takes $O(n)$ time. Thus the overall running time of the algorithm is $O(\log n)$

\question{3}{Square vs Multiply}


\question{4}{Basic Probability}

\part{a}

\part{b}

\question{5}{Array Sums}

\algorithm: 

\correctness: 

\runtime: 

\end{document}
