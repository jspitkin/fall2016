% HW Template for CS 6150, taken from https://www.cs.cmu.edu/~ckingsf/class/02-714/hw-template.tex
%
% You don't need to use LaTeX or this template, but you must turn your homework in as
% a typeset PDF somehow.
%
% How to use:
%    1. Update your information in section "A" below
%    2. Write your answers in section "B" below. Precede answers for all 
%       parts of a question with the command "\question{n}{desc}" where n is
%       the question number and "desc" is a short, one-line description of 
%       the problem. There is no need to restate the problem.
%    3. If a question has multiple parts, precede the answer to part x with the
%       command "\part{x}".
%    4. If a problem asks you to design an algorithm, use the commands
%       \algorithm, \correctness, \runtime to precede your discussion of the 
%       description of the algorithm, its correctness, and its running time, respectively.
%    5. You can include graphics by using the command \includegraphics{FILENAME}
%
\documentclass[11pt]{article}
\usepackage{amsmath,amssymb,amsthm}
\usepackage{graphicx}
\usepackage[margin=1in]{geometry}
\usepackage{fancyhdr}
\usepackage{algorithm}
\usepackage{algpseudocode}
\usepackage{pifont}
\setlength{\parindent}{0pt}
\setlength{\parskip}{5pt plus 1pt}
\setlength{\headheight}{13.6pt}
\newcommand\question[2]{\vspace{.25in}\hrule\textbf{#1: #2}\vspace{.5em}\hrule\vspace{.10in}}
\renewcommand\part[1]{\vspace{.10in}\textbf{(#1)}}
\newcommand\algorith{\vspace{.10in}\textbf{Algorithm: }}
\newcommand\correctness{\vspace{.10in}\textbf{Correctness: }}
\newcommand\runtime{\vspace{.10in}\textbf{Running time: }}
\pagestyle{fancyplain}
\lhead{\textbf{\NAME\ (\UID)}}
\chead{\textbf{HW\HWNUM}}
\rhead{CS 6350, \today}
\begin{document}\raggedright
%Section A==============Change the values below to match your information==================
\newcommand\NAME{Jake Pitkin}  % your name
\newcommand\UID{u0891770}     % your utah UID
\newcommand\HWNUM{6}              % the homework number
%Section B==============Put your answers to the questions below here=======================

\question{1}{Probabilities}

\begin{equation}
\setlength\fboxsep{0.25cm}
\setlength\fboxrule{0.4pt}
\boxed{Independent \ Events- \ P(A \ \cap \ B) = P(A)P(B)}
\end{equation} 

\begin{equation}
\setlength\fboxsep{0.25cm}
\setlength\fboxrule{0.4pt}
\boxed{Rule \ of \ Multiplication- \ P(A \ \cap \ B) = P(A)P(B | A)}
\end{equation} 

\begin{equation}
\setlength\fboxsep{0.25cm}
\setlength\fboxrule{0.4pt}
\boxed{Rule \ of \ Addition- \ P(A \ \cup \ B) = P(A) + P(B) - P(A)P(B|A)}
\end{equation} 

\begin{equation}
\setlength\fboxsep{0.25cm}
\setlength\fboxrule{0.4pt}
\boxed{Binomial \ Distribution-  {n \choose k}p^k(1-p)^{n-k}}
\end{equation} 

\part{1} Given $P(A_1) = P(A_2) = P(A_1 | A_2) = \frac{1}{2}$, we want to prove that $A_1$ and $A_2$ are independent events.

Events $A_1$ and $A_2$ are independent if and only if Eq. (1) is satisfied.
$$P(A_2 \cap A_1) = P(A_2)P(A_1)$$
 We can use Eq. (2) to restate the LHS of Eq. (1) in terms of probabilities we are given.
$$P(A_2)P(A_1 | A_2) = P(A_2)P(A_1)$$
$$\frac{1}{2} * \frac{1}{2} = \frac{1}{2} * \frac{1}{2}$$

By showing Eq. (1) is satisfied, we have proven $A_1$ and $A_2$ are independent events.

\part{2}

\part{3} Let $X$ be a random variable representing the top of the six-sided die toss. The dice is a fair dice so we know $P(X=1) = P(X=2) = P(X=3) = P(X=4) = P(X=5) = P(X=6) = \frac{1}{6}$. There are six possible events and the total probability of exactly two heads after $n$ coin tosses is the sum of the probability of each of the six events happening. 

$$\sum_{i = 1}^6 P(X=i) * B(n = i, k = 2, p = 0.5)$$

Where $B(n, k, p)$ represents the binomial distribution from Eq. (4). This is the probability that given $n$ trials, there are exactly $k$ successes if the probability of success is $p$ where $O \leq p \leq 1$ and $k \leq n$. Let $Y$ be a random variable representing the exact number of heads after $n$ coin tosses. Note that the probability of getting exactly two heads when only tossing one coin is 0.

$$P(Y = 2) = \sum_{i = 1}^6 P(X=i) * B(n = i, k = 2, p = 0.5)$$
$$P(Y = 2) = \frac{1}{6}\sum_{i = 1}^6 B(n = i, k = 2, p = 0.5)$$
$$P(Y = 2) = \frac{1}{6}(0 + \frac{1}{4} + \frac{3}{8} + \frac{6}{16} + \frac{10}{32} + \frac{15}{64})$$
$$P(Y = 2) = \frac{33}{128} = 0.2578$$

Thus this is the probability of getting exactly 2 heads after $n$ coin flips, where $n$ is the result of a fair six-sided die toss.

\framebox[1.2\width]{$P(heads = 2) = \frac{33}{128} = 0.2578$}

\part{4} We want to prove that if $P(A_1) = a_1$ and $P(A_2) = a_2$ then $P(A_1|A_2) \geq \frac{a_1 + a_2 -1}{a_2}$.

\textit{Proof:} we begin with Eq. (3) which is the rule for union of two events.

$$P(A_2 \cup A_1) = P(A_2) + P(A_1) - P(A_2)P(A_1 | A_2)$$

$P(A_2 \cup A_1)$ is a probability so we know it has a upper bound of 1.

$$1 \geq P(A_2 \cup A_1) = P(A_2) + P(A_1) - P(A_2)P(A_1 | A_2)$$
$$1 \geq P(A_2) + P(A_1) - P(A_2)P(A_1 | A_2)$$

Rearranging terms and multiplying both sides by -1.

$$\frac{1 - P(A_2) - P(A_1)}{P(A_2)} \geq -P(A_1 | A_2)$$
$$\frac{P(A_1) + P(A_2) - 1}{P(A_2)} \leq P(A_1 | A_2)$$

Replacing $P(A_1) = a_1$ and $P(A_2) = a_2$ on the LHS of the inequality.

\begin{equation}
\setlength\fboxsep{0.25cm}
\setlength\fboxrule{0.4pt}
\boxed{P(A_1 | A_2) \geq \frac{a_1 + a_2 - 1}{a_2}}
\end{equation} 

Thus arriving at the original inequality and proving it's correctness.

\part{5a}

\end{document}
