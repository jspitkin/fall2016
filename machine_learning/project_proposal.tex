% HW Template for CS 6150, taken from https://www.cs.cmu.edu/~ckingsf/class/02-714/hw-template.tex
%
% You don't need to use LaTeX or this template, but you must turn your homework in as
% a typeset PDF somehow.
%
% How to use:
%    1. Update your information in section "A" below
%    2. Write your answers in section "B" below. Precede answers for all 
%       parts of a question with the command "\question{n}{desc}" where n is
%       the question number and "desc" is a short, one-line description of 
%       the problem. There is no need to restate the problem.
%    3. If a question has multiple parts, precede the answer to part x with the
%       command "\part{x}".
%    4. If a problem asks you to design an algorithm, use the commands
%       \algorithm, \correctness, \runtime to precede your discussion of the 
%       description of the algorithm, its correctness, and its running time, respectively.
%    5. You can include graphics by using the command \includegraphics{FILENAME}
%
\documentclass[11pt]{article}
\usepackage{amsmath,amssymb,amsthm}
\usepackage{graphicx}
\usepackage[margin=1in]{geometry}
\usepackage{fancyhdr}
\usepackage{algorithm}
\usepackage{algpseudocode}
\usepackage{pifont}
\setlength{\parindent}{0pt}
\setlength{\parskip}{5pt plus 1pt}
\setlength{\headheight}{13.6pt}
\newcommand\question[2]{\vspace{.25in}\hrule\textbf{#1#2}\vspace{.5em}\hrule\vspace{.10in}}
\renewcommand\part[1]{\vspace{.10in}\textbf{(#1)}}
\newcommand\algorith{\vspace{.10in}\textbf{Algorithm: }}
\newcommand\correctness{\vspace{.10in}\textbf{Correctness: }}
\newcommand\runtime{\vspace{.10in}\textbf{Running time: }}
\pagestyle{fancyplain}
\lhead{\textbf{\NAME\ (\UID)}}
\chead{\textbf{Project Proposal}}
\rhead{CS 6350, \today}
\begin{document}\raggedright
%Section A==============Change the values below to match your information==================
\newcommand\NAME{Jake Pitkin}  % your name
\newcommand\UID{u0891770}     % your utah UID
\newcommand\HWNUM{2}              % the homework number
%Section B==============Put your answers to the questions below here=======================

For my team name I will be using my last name: Pitkin (creative I know).

\question{}{Type of Project - Competitive Project}

I will be participating in the competitive project option. I think friendly competition will be a fun motivator to tune and produce a quality final project.

\question{}{Algorithms - Linear Classifiers and Naive Bayes}

I enjoyed homework 2 and working with linear classifiers. Currently, for the final project I would like to work with one of the more advanced variants of the Perceptron algorithm.

I also enjoy statistics in particular Bayesian statistics. Using Naive Bayes for classification is another idea I have for the final project.

As we get more details about the competitive project and as we learn more algorithms, my vision for my final project will grow stronger.

\question{}{Experiments - Hyperparameters and Algorithm Variants}

We have seen that k-fold cross validation can be a powerful tool for choosing hyperparameters. I will be implementing this for my final project as this is a strong tool in the art of choosing strong hyperparameters.

In addition, we saw in homework 2 that general algorithms like Perceptron has many variants. These variants perform differently depending on the the data set being classified. For the final project, I will be experimenting with different variants in an attempt to maximize my classification accuracy.


\end{document}
