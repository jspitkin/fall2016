\documentclass[11pt]{article}
\usepackage{amsmath,amssymb,amsthm}
\usepackage{graphicx}
\usepackage{tikz}
\usetikzlibrary{arrows, quotes, trees}
\usepackage[margin=1in]{geometry}
\usepackage{fancyhdr}
\setlength{\parindent}{0pt}
\setlength{\parskip}{5pt plus 1pt}
\setlength{\headheight}{13.6pt}
\newcommand\question[2]{\vspace{.25in}\hrule\textbf{#1: #2}\vspace{.5em}\hrule\vspace{.10in}}
\renewcommand\part[1]{\vspace{.10in}\textbf{(#1)}}
\pagestyle{fancyplain}
\lhead{\textbf{\NAME\ (\UID)}}
\chead{\textbf{HW\HWNUM}}
\rhead{CS 6350, \today}

\tikzstyle{block} = [rectangle, draw, fill=white!20,
    text width=10em, text centered, rounded corners, minimum height=2em]
\tikzstyle{line} = [draw, very thick, color=black!50, -latex']
\tikzstyle{leaf} = [draw=none, fill=none,
    text width=10em, text centered, minimum height=1em]

\begin{document}\raggedright

\newcommand\NAME{Jake Pitkin}
\newcommand\UID{u0891770}
\newcommand\HWNUM{1}

\question{1}{Decision tree}
\part{1} Representing Boolean functions as decision trees.

\qquad \part{a} $(x_1 \lor x_2) \land x_3$

\begin{tikzpicture}[
box/.style = {rectangle, draw, align=center},
level distance = 18mm,
level 1/.style = {sibling distance=66mm},
level 2/.style = {sibling distance=32mm},
level 4/.style = {sibling distance=44mm},
edge from parent/.style = {draw, -latex'},
edge from parent fork down
                        ]
\node [block] {$x_3$}
    child{ node [leaf] {0} edge from parent node[left] {0} edge from parent node[right] {1}}
		child{ node [block] {$x_1$} 
			child{ node [block] {$x_2$}}
	};

\end{tikzpicture}

\qquad \part{b} $(x_1 \land x_2) \ xor\ (\neg x_1 \lor x_3)$

\qquad \part{c} The 2-of-3 function defined as follows: at least 2 of $\{x_1, x_2, x_3\}$ should be true for the \hspace*{13 mm} output to be true.

\part{2} Pok\'{e}mon Go decision tree to determine whether a Pok\'{e}mon can be caught.

\qquad \part{a}

\qquad \part{b}

\qquad \part{c}

\qquad \part{d}

\qquad \part{e}

\qquad \part{f}

\qquad \part{g}

\part{3} Using the Gini measure with the ID3 algorithm.

\qquad \part{a}

\qquad \part{b}

\question{2}{Linear Classifiers}

\part{1}

\part{2}

\part{3}

\end{document}
